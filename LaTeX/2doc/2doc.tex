\documentclass{article}
\usepackage {amsmath}

\title {Segundo documento de \LaTeX}
\author {Marcela De La Cruz}
\date {\today}
\begin {document}
\maketitle
Segundo documento de LaTeX 


\section{Ecuaci\'on en texto}
Entorno de ecuaci\'on dentro de la l\'inea de texto: $\sum^{\infty}_{n=1}$
con el s\'imbolo \^\ se escriben los super\'indices y con \_ los sub\'indices

\section{Entornos de ecuaci\'on}

Hay varios tipos de entornos de ecuaci\'on. Los m\'as importantes son: 
\begin{enumerate}
\item \emph {equation*}
\begin{enumerate}
	\item \emph{equiation} simple
	\item \emph{split equation} 
\end{enumerate}
\item \emph {align}
\end{enumerate}


\begin{equation}
\frac{d \hat{\theta}}{dt} = \sum^{n}_{k{t}=1} C^{*}_{n}(k_{t})
\end{equation}

\begin{equation*}
\frac{d \hat{\theta}}{dt} = \sum^{n}_{k{t}=1} C^{*}_{n}(k_{t})
\end{equation*}

\begin{align}
P_{c}=& \pi D \nonumber \\
=& 2 \pi r
\end{align}

\begin{equation}
\begin{split}
P_{c}=& \pi D \\
=& 2 \pi r
\end{split}
\end{equation}

\begin{subequations}	
\begin{align}
P_{c}=& \pi D \\
=& 2 \pi r
\end{align}
\end{subequations}

\end{document}
